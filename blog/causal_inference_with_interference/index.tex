% Options for packages loaded elsewhere
% Options for packages loaded elsewhere
\PassOptionsToPackage{unicode}{hyperref}
\PassOptionsToPackage{hyphens}{url}
\PassOptionsToPackage{dvipsnames,svgnames,x11names}{xcolor}
%
\documentclass[
  letterpaper,
  DIV=11,
  numbers=noendperiod]{scrartcl}
\usepackage{xcolor}
\usepackage{amsmath,amssymb}
\setcounter{secnumdepth}{-\maxdimen} % remove section numbering
\usepackage{iftex}
\ifPDFTeX
  \usepackage[T1]{fontenc}
  \usepackage[utf8]{inputenc}
  \usepackage{textcomp} % provide euro and other symbols
\else % if luatex or xetex
  \usepackage{unicode-math} % this also loads fontspec
  \defaultfontfeatures{Scale=MatchLowercase}
  \defaultfontfeatures[\rmfamily]{Ligatures=TeX,Scale=1}
\fi
\usepackage{lmodern}
\ifPDFTeX\else
  % xetex/luatex font selection
\fi
% Use upquote if available, for straight quotes in verbatim environments
\IfFileExists{upquote.sty}{\usepackage{upquote}}{}
\IfFileExists{microtype.sty}{% use microtype if available
  \usepackage[]{microtype}
  \UseMicrotypeSet[protrusion]{basicmath} % disable protrusion for tt fonts
}{}
\makeatletter
\@ifundefined{KOMAClassName}{% if non-KOMA class
  \IfFileExists{parskip.sty}{%
    \usepackage{parskip}
  }{% else
    \setlength{\parindent}{0pt}
    \setlength{\parskip}{6pt plus 2pt minus 1pt}}
}{% if KOMA class
  \KOMAoptions{parskip=half}}
\makeatother
% Make \paragraph and \subparagraph free-standing
\makeatletter
\ifx\paragraph\undefined\else
  \let\oldparagraph\paragraph
  \renewcommand{\paragraph}{
    \@ifstar
      \xxxParagraphStar
      \xxxParagraphNoStar
  }
  \newcommand{\xxxParagraphStar}[1]{\oldparagraph*{#1}\mbox{}}
  \newcommand{\xxxParagraphNoStar}[1]{\oldparagraph{#1}\mbox{}}
\fi
\ifx\subparagraph\undefined\else
  \let\oldsubparagraph\subparagraph
  \renewcommand{\subparagraph}{
    \@ifstar
      \xxxSubParagraphStar
      \xxxSubParagraphNoStar
  }
  \newcommand{\xxxSubParagraphStar}[1]{\oldsubparagraph*{#1}\mbox{}}
  \newcommand{\xxxSubParagraphNoStar}[1]{\oldsubparagraph{#1}\mbox{}}
\fi
\makeatother


\usepackage{longtable,booktabs,array}
\usepackage{calc} % for calculating minipage widths
% Correct order of tables after \paragraph or \subparagraph
\usepackage{etoolbox}
\makeatletter
\patchcmd\longtable{\par}{\if@noskipsec\mbox{}\fi\par}{}{}
\makeatother
% Allow footnotes in longtable head/foot
\IfFileExists{footnotehyper.sty}{\usepackage{footnotehyper}}{\usepackage{footnote}}
\makesavenoteenv{longtable}
\usepackage{graphicx}
\makeatletter
\newsavebox\pandoc@box
\newcommand*\pandocbounded[1]{% scales image to fit in text height/width
  \sbox\pandoc@box{#1}%
  \Gscale@div\@tempa{\textheight}{\dimexpr\ht\pandoc@box+\dp\pandoc@box\relax}%
  \Gscale@div\@tempb{\linewidth}{\wd\pandoc@box}%
  \ifdim\@tempb\p@<\@tempa\p@\let\@tempa\@tempb\fi% select the smaller of both
  \ifdim\@tempa\p@<\p@\scalebox{\@tempa}{\usebox\pandoc@box}%
  \else\usebox{\pandoc@box}%
  \fi%
}
% Set default figure placement to htbp
\def\fps@figure{htbp}
\makeatother


% definitions for citeproc citations
\NewDocumentCommand\citeproctext{}{}
\NewDocumentCommand\citeproc{mm}{%
  \begingroup\def\citeproctext{#2}\cite{#1}\endgroup}
\makeatletter
 % allow citations to break across lines
 \let\@cite@ofmt\@firstofone
 % avoid brackets around text for \cite:
 \def\@biblabel#1{}
 \def\@cite#1#2{{#1\if@tempswa , #2\fi}}
\makeatother
\newlength{\cslhangindent}
\setlength{\cslhangindent}{1.5em}
\newlength{\csllabelwidth}
\setlength{\csllabelwidth}{3em}
\newenvironment{CSLReferences}[2] % #1 hanging-indent, #2 entry-spacing
 {\begin{list}{}{%
  \setlength{\itemindent}{0pt}
  \setlength{\leftmargin}{0pt}
  \setlength{\parsep}{0pt}
  % turn on hanging indent if param 1 is 1
  \ifodd #1
   \setlength{\leftmargin}{\cslhangindent}
   \setlength{\itemindent}{-1\cslhangindent}
  \fi
  % set entry spacing
  \setlength{\itemsep}{#2\baselineskip}}}
 {\end{list}}
\usepackage{calc}
\newcommand{\CSLBlock}[1]{\hfill\break\parbox[t]{\linewidth}{\strut\ignorespaces#1\strut}}
\newcommand{\CSLLeftMargin}[1]{\parbox[t]{\csllabelwidth}{\strut#1\strut}}
\newcommand{\CSLRightInline}[1]{\parbox[t]{\linewidth - \csllabelwidth}{\strut#1\strut}}
\newcommand{\CSLIndent}[1]{\hspace{\cslhangindent}#1}



\setlength{\emergencystretch}{3em} % prevent overfull lines

\providecommand{\tightlist}{%
  \setlength{\itemsep}{0pt}\setlength{\parskip}{0pt}}



 


\KOMAoption{captions}{tableheading}
\makeatletter
\@ifpackageloaded{tcolorbox}{}{\usepackage[skins,breakable]{tcolorbox}}
\@ifpackageloaded{fontawesome5}{}{\usepackage{fontawesome5}}
\definecolor{quarto-callout-color}{HTML}{909090}
\definecolor{quarto-callout-note-color}{HTML}{0758E5}
\definecolor{quarto-callout-important-color}{HTML}{CC1914}
\definecolor{quarto-callout-warning-color}{HTML}{EB9113}
\definecolor{quarto-callout-tip-color}{HTML}{00A047}
\definecolor{quarto-callout-caution-color}{HTML}{FC5300}
\definecolor{quarto-callout-color-frame}{HTML}{acacac}
\definecolor{quarto-callout-note-color-frame}{HTML}{4582ec}
\definecolor{quarto-callout-important-color-frame}{HTML}{d9534f}
\definecolor{quarto-callout-warning-color-frame}{HTML}{f0ad4e}
\definecolor{quarto-callout-tip-color-frame}{HTML}{02b875}
\definecolor{quarto-callout-caution-color-frame}{HTML}{fd7e14}
\makeatother
\makeatletter
\@ifpackageloaded{caption}{}{\usepackage{caption}}
\AtBeginDocument{%
\ifdefined\contentsname
  \renewcommand*\contentsname{Table of contents}
\else
  \newcommand\contentsname{Table of contents}
\fi
\ifdefined\listfigurename
  \renewcommand*\listfigurename{List of Figures}
\else
  \newcommand\listfigurename{List of Figures}
\fi
\ifdefined\listtablename
  \renewcommand*\listtablename{List of Tables}
\else
  \newcommand\listtablename{List of Tables}
\fi
\ifdefined\figurename
  \renewcommand*\figurename{Figure}
\else
  \newcommand\figurename{Figure}
\fi
\ifdefined\tablename
  \renewcommand*\tablename{Table}
\else
  \newcommand\tablename{Table}
\fi
}
\@ifpackageloaded{float}{}{\usepackage{float}}
\floatstyle{ruled}
\@ifundefined{c@chapter}{\newfloat{codelisting}{h}{lop}}{\newfloat{codelisting}{h}{lop}[chapter]}
\floatname{codelisting}{Listing}
\newcommand*\listoflistings{\listof{codelisting}{List of Listings}}
\makeatother
\makeatletter
\makeatother
\makeatletter
\@ifpackageloaded{caption}{}{\usepackage{caption}}
\@ifpackageloaded{subcaption}{}{\usepackage{subcaption}}
\makeatother
\usepackage{bookmark}
\IfFileExists{xurl.sty}{\usepackage{xurl}}{} % add URL line breaks if available
\urlstyle{same}
\hypersetup{
  pdftitle={Causal Inference With Interference},
  colorlinks=true,
  linkcolor={blue},
  filecolor={Maroon},
  citecolor={Blue},
  urlcolor={Blue},
  pdfcreator={LaTeX via pandoc}}


\title{Causal Inference With Interference}
\author{}
\date{2025-12-01}
\begin{document}
\maketitle


In a causal experiment we would like to measure the impact of a
treatment on a quantity of interest. To fix ideas, suppose we have a
social media site and are considering an update on the messaging
function. The goal might be to increase user messaging frequency or even
to improve the quality of messages as measured by the overall length of
each message. In the classical sense, there is no way to practically
isolate users in a way that would satisfy the SUTVA assumption, hence
the need to update the theoretical foundations causal experiments are
based upon.

Classic causal testing typically relies on an assumption of no
interference between individuals (for example, the SUTVA framework).
With the no-interference assumption, the outcomes of each individual are
unaffected by the treatment of each other individual.

Social media created a need for causal testing theory and methods that
flips that assumptions and conversely handles the case of dense
interactions between individuals. From the perspective of the classical
theory there would be no guarantee such methodolody would exist, which
makes the discoveries of the early social media days a remarkable story.

Background for this note includes the following \ldots{}

\begin{tcolorbox}[enhanced jigsaw, bottomrule=.15mm, toprule=.15mm, colback=white, rightrule=.15mm, toptitle=1mm, arc=.35mm, titlerule=0mm, colbacktitle=quarto-callout-note-color!10!white, coltitle=black, title=\textcolor{quarto-callout-note-color}{\faInfo}\hspace{0.5em}{Estimator}, colframe=quarto-callout-note-color-frame, leftrule=.75mm, breakable, opacityback=0, bottomtitle=1mm, opacitybacktitle=0.6, left=2mm]

In the language of statistics, an esimator is a formula or rule employed
to estimate a quantity of interest based on observed data. An estimator
is said to be unbiased if it's expectation is equal to the quantity of
interest.

Familiar examples include the Expectation and Variance estimators for a
set of observations \((X_1,..,X_n)\) of a random variable \(X\) with
mean \(\mu\) and variance \(\sigma^2\).

\textbf{Expectation Estimator}

\[
\bar X_n = \frac{1}{n} \sum_{i=1}^n X_i
\] The estimator is unbiased since \[
\mathbb{E} [\bar X_n] = \mu
\]

It is useful to find (or at least bound above) the variance of the
estimator, since that guarantees a good estimate if it can be shown to
be small. The variance is \[
Var(\bar X_n)  = \mathbb{E}[(\bar X_n - \mathbb{E}[\bar X_n])^2] = \frac{1}{n}\sigma^2
\] so the variance tends to zero with a large sample, assuring a quality
estimator that is unbiased and converges to the quantity of interest.

\textbf{Variance Estimator}

\[
S_n^2 = \frac{1}{n-1}\sum_{i=1}^n (X_i - \bar X_n)^2
\] \[
\mathbb{E} [S_n^2] = \sigma^2
\]

The variance is somewhat more complicated, \[
Var(S^2_n) = \frac1n \left( \mathbb{E}[(X - \mu)^4] -\frac{n-3}{n-1}\sigma^4 \right)
\] nevertheless still tends to zero with a large sample.

Note we did not use the estimator \[
\tilde S_n^2 = \frac{1}{n}\sum_{i=1}^n (X_i - \bar X_n)^2
\] as this woule have expectation \[
\mathbb{E} [\tilde S_n^2] = \frac{n-1}{n}\sigma^2
\] and would be a biased estimator.

\end{tcolorbox}

\begin{tcolorbox}[enhanced jigsaw, bottomrule=.15mm, toprule=.15mm, colback=white, rightrule=.15mm, toptitle=1mm, arc=.35mm, titlerule=0mm, colbacktitle=quarto-callout-note-color!10!white, coltitle=black, title=\textcolor{quarto-callout-note-color}{\faInfo}\hspace{0.5em}{Horvitz Thompson Estimator}, colframe=quarto-callout-note-color-frame, leftrule=.75mm, breakable, opacityback=0, bottomtitle=1mm, opacitybacktitle=0.6, left=2mm]

Let \$\mathcal{K} = \{1,2,\ldots,N\} \$ be a finite population. For each
\(i\in\mathcal{K}\), let \(Y_i\) be an observable of unit \(i\). Our
goal is to estimate \[
Y = \sum_{i=1}^N Y_i
\] the total of interest. And \[
\tau = \frac{1}{N}  \sum_{i=1}^N Y_i
\] the average of the population.

To construct the estimator we take a random sample of \(n< N\) units
from \(\mathcal{K}\). Any sampling method is allowed, but the obtained
sample must have distinct units. Let \(\pi_i\) be the probability the
\(i^{th}\) unit is included in the sample (assume \(\pi_i > 0\)). Denote
the random sample by \(\bf{S} = \{i_1,...,i_n\} \subset \mathcal{S}\).
The Horvitz Thompson esimator is \[
\hat Y_{HT} = \sum_{i\in \bf{S}} \frac{Y_i}{\pi_i} = \sum_{i=1}^N \frac{Y_i 1_{\{i\in \bf S\}}}{\pi_i}
\] and \[
\hat \tau_{HT} = \frac{1}{N} \hat Y_{HT} 
\]

The estimator is unbiased \[
    \mathbb{E}[\hat Y_{HT}] = Y.
\] \[
    \mathbb{E}[\hat \tau_{HT}] = \tau.
\]

To state the variance of \(\hat Y_{HT}\), define \(\pi_{ij}\) as the
probability that both \(i\) and \(j\) belong to \(\bf{S}\). The variance
of Horvitz Thompson is then \[
Var(\hat Y_{HT}) 
    = 
        \sum_{i=1}^N \sum_{j=1}^{N} \frac{\pi_{ij} - \pi_i\pi_j}{\pi_i\pi_j} Y_i Y_j
\] it follows from the definition that \(\pi_{ii}\) = \(\pi_i\). The
variance of tha average estimator is \[
Var(\hat \tau_{HT}) 
    = 
        \frac{1}{N^2}\sum_{i=1}^N \sum_{j=1}^{N} \frac{\pi_{ij} - \pi_i\pi_j}{\pi_i\pi_j} Y_i Y_j
\]

As we want to assure that the Horwitz Thompson estimator converges well
to the quantity of interest, controlling the variance is key. It is not
hard to see the variance has an upper bound, \[
Var(\hat Y_{HT}) \leq \sum_{i,j=1}^N \left(\frac{1}{\pi_i^{1/2}\pi_j^{1/2}} - 1\right) Y_i Y_j
\] but if some of the \(\pi_i\) are small this raises the small
denominator problem. So care would have to be taken in this case. This
danger can be mollified through a modification of the HTE, the Hajek
estimator. (Also note that as the sample becomes larger, the \(\pi_i\)
become larger, better controlling the variance.)

\textbf{Example}

Consider a collection of cities with populations greater than some lower
bound \(b\) in some state. Let the cities be enumerated by
\(\mathcal{K} =  \{1,2,..,N\}\). Let \(Y_i\) be the total number of
hotels in the ith city.

Sample \(n\) units from \(\mathcal{K}\) by Simple Random Sampling
Without Replacement. The inclusion probabilities are \[
\pi_i = n/N
\] \[
\pi_{ij} = \frac{n(n-1)}{N(N-1)}
\]

The estimate of the average number of hotels in large cities would be
given by \(\hat Y_{HT}\) after sampling \[
\hat Y_{HT} = \sum_{i\in \bf{S}} \frac{Y_i}{\pi_i}
 = \frac{N}{n}\sum_{i\in \bf{S}} Y_i
\] with variance \[
Var(\hat Y_{HT}) = N^2 \frac{1 - n/N}{n} S^2
\] where \(S^2\) is the population variance. The variance does not tend
to zero even as m tends to infinity. But, for the Horwitz Thompson
average, \[
Var(\hat \tau_{HT}) = \frac{1 - n/N}{n} S^2
\] Making a simple assumption \(n \propto N^\alpha\) for
\(0 < \alpha < 1\) or \(n = \beta N\) for \(0 < \beta < 1\), the
variance of the estimator of the average tends to zero.

\end{tcolorbox}

Enumerate the users \(\mathcal{V} = \{v_1,v_2,...,v_N\}\). Typically,
the network is modeled as a graph
\(\mathcal{G} = (\mathcal{V},\mathcal{E})\). In our example two units
may be connected by an edge if they are freinds. For the sake of the
experiment, each user is assigned to one of two groups \(z_i = 0,1\),
with \(0\) being the control and \(1\) being the treatment.

With an assignment of \(z = (z_1,z_2,..,z_N)\) the total of the quantity
of interest is \[
Y(z) = \sum_{i=1}^N Y_i(z)
\] where \(Y_i\) is the quantity of interest for the \(i^{th}\)
individual. For our messaging example, \(Y_i\) might be the total number
of messages the user sent within a two week period. Notice at this point
we are allowing that each individual's behavior depends on the entire
assignment vector!

The ulimate goal is to estimate \[
\tau(\vec{1},\vec{0}) = \frac1N\sum_{i=1}^N \left(Y_i(\vec{1}) - Y_i(\vec{0}) \right)
\label{eq_ate}
\] the Average Treatment Effect (ATE). Here \(\vec{1}\) (\(\vec{0}\)) is
the assignment of \(1\)s (\(0\)s) to all individuals. Of course there is
no world where we can simultaneously assign all units both 1 and 0,
hence the goal of our study.

It would be hard to make it very far without making some kind of
assumptions on the network effects. How can we make conclusions about
\(\tau(\vec{1},\vec{0})\) if we can't isolate the assignment groups? The
earliest paper I found in this direction is (Ugander et al. 2013). The
authors make the assumption of a \textbf{\emph{net exposure condition}}
which effectively assumes that if sufficiently many agents of an
individuals neighborhood belong to a given assignment group, the
individual will behave as if the entire network has that assignment.
That is, for indivual \(v\in\mathcal{V}\), suppose there are \(k_i\)
individuals in \(A_{1,r}(v)\) assigned \(z_j = 1 (0)\), then we can
assume \(i\) behaves as if \(z = \vec{1} (\vec{0})\). Explicit choices
of \(k_i\) can vary, but the authors consider \(k_i\) the size of the
neighborhood \(k_i = |A_{1,r}(v)|\), or a proportion of the neighborhood
\(k_i = \kappa |A_{1,r}(v)|\), or simply a constant minimum over the
network \(k_i = \kappa\).

The Ugander paper is notable because they tackle the problem of a
globally connected network of individuals. Compare this to the more
mature literature of estimating the ATE when there are groups of
interacting agents (Hudgens and Halloran 2008). In that paper, the
population is stratified into groups, and interference is limited to
individuals among the same group. For example, a group might be all
students in a particular elementary school. The analogy of the net
exposure condition, is the assumption that any two assignments of a
fixed group are equivalent if they have the same number of individuals
assigned to 1 and 0 respectively.

\begin{tcolorbox}[enhanced jigsaw, bottomrule=.15mm, toprule=.15mm, colback=white, rightrule=.15mm, toptitle=1mm, arc=.35mm, titlerule=0mm, colbacktitle=quarto-callout-note-color!10!white, coltitle=black, title=\textcolor{quarto-callout-note-color}{\faInfo}\hspace{0.5em}{Hudgens and Halloran, 2008}, colframe=quarto-callout-note-color-frame, leftrule=.75mm, breakable, opacityback=0, bottomtitle=1mm, opacitybacktitle=0.6, left=2mm]

The Hudgens and Halloran paper outlines detailed experiment designs. The
goal is to measure quantities like (\textbf{eq\_ate?})

\end{tcolorbox}

papers (Hudgens and Halloran 2008), (Aronow and Samii 2017), (Ugander et
al. 2013), (Eckles, Karrer, and Ugander 2017)

\phantomsection\label{refs}
\begin{CSLReferences}{1}{0}
\bibitem[\citeproctext]{ref-Aronow2017Estimating}
Aronow, Peter M., and Cyrus Samii. 2017. {``Estimating Average Causal
Effects Under General Interference, with Application to a Social Network
Experiment.''} \emph{The Annals of Applied Statistics} 11 (4): 1912--47.
\url{http://www.jstor.org/stable/26362172}.

\bibitem[\citeproctext]{ref-Eckles2017Design}
Eckles, Dean, Brian Karrer, and Johan Ugander. 2017. {``Design and
Analysis of Experiments in Networks: Reducing Bias from Interference.''}
\emph{Journal of Causal Inference} 5 (1): 20150021.

\bibitem[\citeproctext]{ref-Hudgens2008Causal}
Hudgens, Michael G., and M. Elizabeth Halloran. 2008. {``Toward Causal
Inference with Interference.''} \emph{Journal of the American
Statistical Association} 103 (482): 832--42.
\url{http://www.jstor.org/stable/27640105}.

\bibitem[\citeproctext]{ref-Ugander2013Graph}
Ugander, Johan, Brian Karrer, Lars Backstrom, and Jon Kleinberg. 2013.
{``Graph Cluster Randomization: Network Exposure to Multiple
Universes.''} In \emph{Proceedings of the 19th ACM SIGKDD International
Conference on Knowledge Discovery and Data Mining}, 329--37. KDD '13.
New York, NY, USA: Association for Computing Machinery.
\url{https://doi.org/10.1145/2487575.2487695}.

\end{CSLReferences}




\end{document}
